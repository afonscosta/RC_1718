\documentclass{llncs}
\usepackage{times}
\usepackage[T1]{fontenc}
\usepackage[utf8]{inputenc}

% Comentar para not MAC Users
%\usepackage[applemac]{inputenc}

\usepackage{a4}
%\usepackage[margin=3cm,nohead]{geometry}
\usepackage{epstopdf}
\usepackage{graphicx}
\usepackage{fancyvrb}
\usepackage{amsmath}
\usepackage{subcaption}
\usepackage[bookmarks=false]{hyperref}
%\renewcommand{\baselinestretch}{1.5}

\begin{document}
\mainmatter
\title{TP4: Redes sem fios (802.11)}

\titlerunning{TP4: Redes sem fios (802.11)}

\author{Diogo Afonso Costa \and Daniel Maia \and Vitor Castro}

\authorrunning{Diogo Afonso Costa \and Daniel Maia \and Vitor Castro}

\institute{
University of Minho, Department of  Informatics, 4710-057 Braga, Portugal\\
e-mail: \{a78034,a77531,a77870\}@alunos.uminho.pt
}

\date{12 de dezembro de 2017}
\bibliographystyle{splncs}

\maketitle
\begin{abstract}

\end{abstract}

\section{Introdução}



\clearpage
\section{Acesso Rádio (Para a trama correspondente 733)}

\subsection{Exercício 1}
\subsubsection{Questão}\rule[-10pt]{0pt}{10pt}\\

Identifique em que frequência do espectro está a operar a rede sem fios, e o canal que corresponde essa frequência.

\subsubsection{Resposta}\rule[-10pt]{0pt}{10pt}\\



\subsubsection{Realização}\rule[-10pt]{0pt}{10pt}\\



\clearpage
\subsection{Exercício 2}
\subsubsection{Questão}\rule[-10pt]{0pt}{10pt}\\

Identifique a versão da norma IEEE 802.11 que está a ser usada.

\subsubsection{Resposta}\rule[-10pt]{0pt}{10pt}\\



\subsubsection{Realização}\rule[-10pt]{0pt}{10pt}\\



\clearpage
\subsection{Exercício 3}
\subsubsection{Questão}\rule[-10pt]{0pt}{10pt}\\

Qual o débito a que foi enviada a trama escolhida? Será que esse débito corresponde ao débito máximo a que a interface WiFi pode operar? Justifique.

\subsubsection{Resposta}\rule[-10pt]{0pt}{10pt}\\



\subsubsection{Realização}\rule[-10pt]{0pt}{10pt}\\



\clearpage

\section{Scanning Passivo e Scanning Ativo}
\subsection{Exercício 4}
\subsubsection{Questão}\rule[-10pt]{0pt}{10pt}\\

Selecione uma trama beacon (cujo número de ordem inclua o seu número de grupo [33]). Esta trama pertence a que tipo de tramas 802.11? Indique o valor dos seus identificadores de tipo e de subtipo. Em que parte concreta do cabeçalho da trama estão especificados (ver anexo)?

\subsubsection{Resposta}\rule[-10pt]{0pt}{10pt}\\



\subsubsection{Realização}\rule[-10pt]{0pt}{10pt}\\



\clearpage
\subsection{Exercício 5}
\subsubsection{Questão}\rule[-10pt]{0pt}{10pt}\\

Liste todos os SSIDs dos APs (Access Points) que estão a operar na vizinhança da STA de captura? Explicite o modo como obteve essa informação. Como sugestão pode construir um filtro de visualização apropriado (tomando como base a resposta da alínea anterior) que lhe permita obter a listagem pretendida.

\subsubsection{Resposta}\rule[-10pt]{0pt}{10pt}\\



\subsubsection{Realização}\rule[-10pt]{0pt}{10pt}\\



\clearpage
\subsection{Exercício 6}
\subsubsection{Questão}\rule[-10pt]{0pt}{10pt}\\

Verifique se está a ser usado o método de detecção de erros (CRC), e se todas as tramas Beacon são recebidas corretamente. Justifique a conveniência em usar detecção de erros neste tipo de redes locais.

\subsubsection{Resposta}\rule[-10pt]{0pt}{10pt}\\



\subsubsection{Realização}\rule[-10pt]{0pt}{10pt}\\



\clearpage
\subsection{Exercício 7}
\subsubsection{Questão}\rule[-10pt]{0pt}{10pt}\\

Para dois dos APs identificados, indique qual é o intervalo de tempo previsto entre tramas beacon consecutivas? (Nota: este valor é anunciado na própria trama beacon). Na prática, a periodicidade de tramas beacon é verificada? Tente explicar porquê.

\subsubsection{Resposta}\rule[-10pt]{0pt}{10pt}\\



\subsubsection{Realização}\rule[-10pt]{0pt}{10pt}\\



\clearpage
\subsection{Exercício 8}
\subsubsection{Questão}\rule[-10pt]{0pt}{10pt}\\

Identifique e registe todos os endereços MAC usados nas tramas beacon enviadas pelos APs. Recorde que o endereçamento está definido no cabeçalho das tramas 802.11, podendo ser utilizados até quatro endereços com diferente semântica. Para uma descrição detalhada da estrutura da trama 802.11, consulte o anexo ao enunciado.

\subsubsection{Resposta}\rule[-10pt]{0pt}{10pt}\\



\subsubsection{Realização}\rule[-10pt]{0pt}{10pt}\\



\clearpage
\subsection{Exercício 9}
\subsubsection{Questão}\rule[-10pt]{0pt}{10pt}\\

As tramas beacon anunciam que o AP pode suportar vários débitos de base assim como vários “extended supported rates”. Indique quais são esses débitos?

\subsubsection{Resposta}\rule[-10pt]{0pt}{10pt}\\



\subsubsection{Realização}\rule[-10pt]{0pt}{10pt}\\



\clearpage
\subsection{Exercício 10}
\subsubsection{Questão}\rule[-10pt]{0pt}{10pt}\\

Estabeleça um filtro Wireshark apropriado que lhe permita visualizar todas as tramas probing request ou probing response, simultaneamente.

\subsubsection{Resposta}\rule[-10pt]{0pt}{10pt}\\



\subsubsection{Realização}\rule[-10pt]{0pt}{10pt}\\




\clearpage
\subsection{Exercício 11}
\subsubsection{Questão}\rule[-10pt]{0pt}{10pt}\\

Identifique um probing request para o qual tenha havido um probing response. Face ao endereçamento usado, indique a que sistemas são endereçadas estas tramas e explique qual o propósito das mesmas?

\subsubsection{Resposta}\rule[-10pt]{0pt}{10pt}\\



\subsubsection{Realização}\rule[-10pt]{0pt}{10pt}\\



\clearpage

\section{Processo de Associação}
\subsection{Exercício 12}
\subsubsection{Questão}\rule[-10pt]{0pt}{10pt}\\

Identifique uma sequência de tramas que corresponda a um processo de associação completo entre a STA e o AP, incluindo a fase de autenticação.

\subsubsection{Resposta}\rule[-10pt]{0pt}{10pt}\\



\subsubsection{Realização}\rule[-10pt]{0pt}{10pt}\\



\clearpage
\subsection{Exercício 13}
\subsubsection{Questão}\rule[-10pt]{0pt}{10pt}\\

Efetue um diagrama que ilustre a sequência de todas as tramas trocadas no processo.

\subsubsection{Resposta}\rule[-10pt]{0pt}{10pt}\\



\subsubsection{Realização}\rule[-10pt]{0pt}{10pt}\\



\clearpage

\section{Transferência de Dados}
\subsection{Exercício 14}
\subsubsection{Questão}\rule[-10pt]{0pt}{10pt}\\

Considere a trama de dados nº1054. Sabendo que o campo Frame Control contido no cabeçalho das tramas 802.11 permite especificar a direccionalidade das tramas, o que pode concluir face à direccionalidade dessa trama, será local à WLAN?

\subsubsection{Resposta}\rule[-10pt]{0pt}{10pt}\\


\subsubsection{Realização}\rule[-10pt]{0pt}{10pt}\\


\clearpage
\subsection{Exercício 15}
\subsubsection{Questão}\rule[-10pt]{0pt}{10pt}\\

Para a trama de dados nº1054, transcreva os endereços MAC em uso, identificando qual o endereço MAC correspondente ao host sem fios (STA), ao AP e ao router de acesso ao sistema de distribuição?

\subsubsection{Resposta}\rule[-10pt]{0pt}{10pt}\\



\subsubsection{Realização}\rule[-10pt]{0pt}{10pt}\\



\clearpage
\subsection{Exercício 16}
\subsubsection{Questão}\rule[-10pt]{0pt}{10pt}\\

Como interpreta a trama nº1060 face à sua direccionalidade e endereçamento MAC?

\subsubsection{Resposta}\rule[-10pt]{0pt}{10pt}\\



\subsubsection{Realização}\rule[-10pt]{0pt}{10pt}\\



\clearpage
\subsection{Exercício 17}
\subsubsection{Questão}\rule[-10pt]{0pt}{10pt}\\

Que subtipo de tramas de controlo são transmitidas ao longo da transferência de dados acima mencionada? Tente explicar porque razão têm de existir (contrariamente ao que acontece numa rede Ethernet.)

\subsubsection{Resposta}\rule[-10pt]{0pt}{10pt}\\



\subsubsection{Realização}\rule[-10pt]{0pt}{10pt}\\



\clearpage
\subsection{Exercício 18}
\subsubsection{Questão}\rule[-10pt]{0pt}{10pt}\\

O uso de tramas Request To Send e Clear To Send, apesar de opcional, é comum para efetuar "pré-reserva" do acesso ao meio quando se pretende enviar tramas de dados, com o intuito de reduzir o número de colisões resultante maioritariamente de STAs escondidas. Para o exemplo acima, verifique se está a ser usada a opção RTS/CTS na troca de dados entre a STA e o AP/Router da WLAN, identificando a direccionalidade das tramas e os sistemas envolvidos.

\subsubsection{Resposta}\rule[-10pt]{0pt}{10pt}\\



\subsubsection{Realização}\rule[-10pt]{0pt}{10pt}\\



\clearpage

\section{Conclusões}

\hspace{5mm} Neste trabalho foi abordada a camada de ligação lógica da pilha OSI e alguns dos seus componentes. 

Primeiramente, procedeu-se à compreensão das tramas \textit{ethernet} que permitiu consolidar bases para analisar as mensagens de ARP e as suas características. A compreensão do protocolo ARP, auxiliada pelos exercícios propostos, permitiu perceber a área em que este protocolo atua e quais as suas consequências.

Por fim, percebeu-se o impacto que têm os diferentes sistemas que constituem a rede. Nomeadamente, o domínio de colisão depende em grande parte da topologia utilizada e caso esta não previna antecipadamente as colisões de tramas na rede, são então utilizados protocolos que tem como objetivo evitar essas mesmas colisões através de diferentes abordagens, como é o caso do protocolo CSMA/CD que apenas transmite quando a rede se encontra desocupada.

%BIBLIOGRAFIA
\bibliographystyle{splncs}
\bibliography{ficheirodebibliografia}

\end{document}
